
\documentclass[10pt,a4paper,ragged2e]{altacv}

%% AltaCV uses the fontawesome and academicon fonts
%% and packages.
%% See texdoc.net/pkg/fontawecome and http://texdoc.net/pkg/academicons for full list of symbols. You MUST compile with XeLaTeX or LuaLaTeX if you want to use academicons.

% Change the page layout if you need to
\geometry{left=1cm,right=9cm,marginparwidth=6.8cm,marginparsep=1.2cm,top=1.25cm,bottom=1.25cm}

% Change the font if you want to, depending on whether
% you're using pdflatex or xelatex/lualatex
\ifxetexorluatex
  % If using xelatex or lualatex:
  \setmainfont{Carlito}
\else
  % If using pdflatex:
  \usepackage[utf8]{inputenc}
  \usepackage[T1]{fontenc}
  \usepackage[default]{lato}
\fi


\definecolor{VividPurple}{HTML}{3E0097}
\definecolor{SlateGrey}{HTML}{2E2E2E}
\definecolor{LightGrey}{HTML}{666666}
\colorlet{heading}{VividPurple}
\colorlet{accent}{VividPurple}
\colorlet{emphasis}{SlateGrey}
\colorlet{body}{LightGrey}


\renewcommand{\itemmarker}{{\small\textbullet}}
\renewcommand{\ratingmarker}{\faCircle}

%% sample.bib contains your publications



\begin{document}
\name{Nicolas Vadkerti}
\tagline{ Administrateur système  }
 
\photo{2.5cm}{Moi}
\personalinfo{%

  \email{nicolas.vadkerti@gmail.com}
    \phone{06 95 69 19 90}
    
  \location{110 grande rue St Michel, 31400 Toulouse}
  
    
    
   \github{github.com/SlaynPool} 
   \permis{Permis B}  
}

\begin{fullwidth}
\makecvheader
\end{fullwidth}
\AtBeginEnvironment{itemize}{\small}
\cvsection[page1sidebar]{Experiences Professionnelles}

\cvevent{}{Administrateur Systeme et Base de donnée}{ 2020- - }{Conseil Départemental , Toulouse}
\begin{itemize}
\item Administration Systèmes Linux CentOS Debian 
\item Gestion de solution de Virtualisation 
\item Administration d'outils de gestion de configuration à grande échelle
\item Automatisation / dévélopement d'outils orienté DevOps
\item Gestion de la production applicative sensible type Nextcloud
\item Administration de service d'annuaire type Active Directory
\item Administration d'outils de Virtualisation d'application (VDI) 
\item Gestion de projet: Migration de 30 annexes du Conseil départemental de Windows 2003 vers Windows 2019.
\item Supervision du Système d'Information
\item Déploiement d'application 
\end{itemize}

\divider
%\cvevent{}{Projet de fin d'année: Drone Open Source}{ 2019-2020 }{IUT de Béziers}
%\begin{itemize}
%\item Conception d'un drone
%\item Développement du Firmware 
%\item Utilisation de Microcontroleurs Arduino \& ESP32 
%\end{itemize}

%\divider

\cvevent{}{Apprenti Administrateur Systeme, DUT}{ 2018-2019 }{CNRS, Route de Mende}
\begin{itemize}
\item Administration Systèmes Linux
\item Administration Réseaux
\item Supervision du Système d'Information
\item Déploiement d'application 
\end{itemize}
\cvevent{}{Etudiant Volontaire Laboratoire de Recherche} { 2018- 2020 }{IUT Béziers}
\begin{itemize}
    \item  Administration Système du Laboratoire
    \item Support projets de Recherche
\end{itemize}

\divider

\cvsection{Formations}
\cvevent{}{Licence Professionnelle Réseau \& Télécomunication, spécialité IoT (Internet Of Things)}{ 2019-2020}{IUT de Béziers}
\cvevent{}{DUT Réseau et Télécomunication}{2017-2019}{IUT de Béziers}
\begin{itemize}
\item 2eme Année réalisée en Apprentissage au sein du service Informatique (SIC) du CNRS, Route de Mende
\end{itemize}
\cvevent{}{ESMA Montpellier}{2016-2017}{Montpellier}
\cvevent{}{Baccalauréat Scientifique}{2016}{Lycée Jean Moulin, Pézenas}
\clearpage
\end{document}
